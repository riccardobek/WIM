\subsection{Navigazione}
Nella pagina è facilmente raggiungibile quanto si sta cercando con pochi click. Ho fatto il test per andare a cercare l'anime "Naruto Shippuden" prima attraverso la barra di ricerca, poi adottando la navigazione da menù e ci sono voluti rispettivamente 2 e 3 click. Attenzione che questo accade quando non è entrata in gioco il problema generale della pagina della pubblicità (vedi \hyperref[Pubblicità]{Sezione \ref{Pubblicità}}).
Tuttavia a seguito del fenomeno del deep linking, ovvero che i motori di ricerca reindirizzano gli utenti all'interno del sito senza passare per la homepage, i click in questione si riducono notevolmente ed entra in gioco un altro aspetto: "Quali informazioni presento delle 6 W?".
Ebbene il sito ha optato per gestire le informazioni interne nel seguente modo:
\begin{itemize}
	\item \textbf{Who:} in ogni pagina viene fornito il logo del sito;
	\item \textbf{What:} cliccando sul logo è possibile raggiungere la homepage e le informazioni della pagina raggiunta sono minimali ma adeguate (trama, categoria, tag, numero episodi ed altro);
	\item \textbf{When:} nelle pagine più interne questo aspetto viene tenuto nella colonna di destra "Ultime notizie";
	\item \textbf{why:} non è presente nessuno slogan o descrizione che permetta di capire perché si è arrivati al sito in questione e per quale motivo dovrei dargli fiducia;
	\item \textbf{How:} nelle pagine interne non viene tenuto nessun tipo di modalità di ricerca, costringendo l'utente a navigare attraverso lo scroll ed eventualmente tornando alla homepage. 
\end{itemize}
L'ultimo aspetto è il Where. Quando si arriva ad una pagina è importante fin da subito aiutare l'utente a crearsi una mappa mentale che gli spieghi, ricordi come è giunto alla pagina e quali sono i percorsi già affrontati.Di seguito vengono presentate le tecniche e si analizza quali di queste viene adottata.

\newpage

\subsubsection{Breadcrumb}
La breadcrumb (letteralmente "briciole di pane") è una tecnica di navigazione usata nelle interfacce utente. Il loro scopo è quello di fornire agli utenti un modo di tener traccia della loro posizione nel sito. 
Esitono tre diverse tecniche:
\begin{itemize}
	\item \textbf{Location:} indica la pagina raggiunta nella gerarchia del sito;
	\item \textbf{Attribute:} mostra la categoria e gli attributi della pagina;
	\item \textbf{Path:} mostra il cammino dell'utente per giungere alla pagina. È dinamico infatti dipende dal cammino dell'utente e usa dei cookie per tenere traccia di tali informazioni.
\end{itemize}
\nomeSito ha optato per non fornire nessuna di queste informazioni di conseguenza l'utente non può crearsi alcuna mappa mentale del sito.

\subsubsection{Link colorati}
Uno delle convenzioni affermatesi da grazie a Natscape è cambiare colore ai link già visitati. In questo sito ciò accade. Quando si passa sopra ad un link questo cambia colore e dopo esservi entrato, se si dovesse ritornare nella pagina di selezione o si incontrasse nuovamente quel link, questo avrebbe un colore diverso. Il problema in questo caso non è il cambio colore, ma invece è il colore selezionato da blu a viola che ha uno stacco poco rilevante e potrebbe confondere.\\
Il cambio di colore però è associato unicamente ai link nel caso delle voci di menù non avviene e per l'assenza di breadcrumb possiamo dire che l'utente può crearsi una mappa mentale ma molto sommaria e poco dettagliata. 

\subsubsection{Back button}
Rispetto ad un link diretto gli utenti prediligono il pulsante back, andando a minimizzare lo sforzo computazionale immediato piuttosto che il tempo. Il pulsante back funziona, tuttavia possiamo dire, come vederemo nella \hyperref[Pubblicità]{Sezione \ref{Pubblicità}}, la presenza di questa rende all'utente l'esperienza nel sito abbastanza frustrante.