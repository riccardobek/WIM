\subsection{Ricerca} \label{Ricerca}
Un problema fondamentale dei siti è far sì che il proprio contenuto sia trovato dagli utenti. La ricerca diviene quindi un fattore fondamentale all'interno del web. Esistono varie modalità di ricerca, prima di tutto ogni sito internet se abbastanza grande dovrebbe avere come servizio la ricerca interna. In questo caso il sito presenta un contenuto molto elevato quindi la modalità di ricerca è essenziale.
\nomeSito ha optato per la ricerca classica nella quale si inseriscono una serie di parole chiave e si preme il pulsante "Cerca" che manda in esecuzione la ricerca di tutti i contenuti collegati a quella parola.
Il problema è che tale tipo di ricerca lo si ritrova unicamente alla homepage. Le altre pagine non presentano questo strumento assai utile. Il menù non si ingrandisce, ma accetta un numero di parole infinito. Inoltre la modalità di ricerca è molto limitata perché costringe un utente a sapere cosa cercare e quindi deve andare in altri siti per decidere cosa guardare. La soluzione migliore sarebbe quella di categorizzare gli anime in base a dei tag, cosa che già avviene quando vengono creati e poi permettere una ricerca in base ad essi. 
Una modalità simile viene adottata a fondo pagina ma risulta scomoda così come è staata ideata. Sarebbe meglio togliere l'effetto grafico che è stato inserito e migliorare la modalità di ricerca statica.
