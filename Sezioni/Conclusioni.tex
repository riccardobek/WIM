\section{Conclusioni}
\begin{center}
	\begin{tabularx}{\textwidth}{|c|X|c|}
		\hline
		\textbf{Sezione} & \textbf{Valutazione e consigli} &\textbf{Voto} \\ \hline
		
		\textbf{Homepage} &	&	\textbf{6}\\
		\hdashline 
		\multirow{5}{0cm}\
		Where 	& \textit{Buono, ma lasciare che la comprensione di dove si è arrivati sia compresa da qualche scritta poco mirata non mi pare una scelta corretta.} &	6\\
		Who		& \textit{Si capisce chi c'è dietro al sito dopo qualche minuto di navigazione nella homepage, ma una pagina dedicata poteva essere molto più efficacie. Essa potrebbe essere la Contacts che però chiamerei: "Contact us"} &	7\\
		Why 	& \textit{Nulla mi fa dire di avere fiducia nel sito. Aggiungerei uno slogan e una descrizione adeguata.} &	3\\
		What 	& \textit{Non mi limiterei a mostrare le ultime uscite darei anche delle informazioni più dettagliate su alcnue di esse. Così come è è un elenco di immagini (e link).} &	5\\
		When 	& \textit{Ottimo così come è. Non mi piace la questione della novità che è pressoché infinita.} &	8\\
		How 	& \textit{Limitata da migliorare.} &	7\\
		\hline
		
		\textbf{Nome} & \textit{La scelta del nome è azzeccata! Non confonde l'utente e trasmette in modo immediato il topic della pagina.} & \textbf{9} \\ \hline
		
		\textbf{Struttura generale} &	&	\textbf{5}\\
		\hdashline 
		\multirow{5}{0cm}\
		Navigazione & \textit{La navigazione della pagina è pessima. Vi sono pagine raggiungibili solo da alcuni link e non da altri. Non viene facilitato in alcun modo la creazione dello schema mentale.} &	2\\
		Menù		& \textit{Semplice e adeguato. Se reso sempre raggiungibile ponendolo in alto dopo una serie di scoll verticale potrebbe essere molto utile.} &	7\\
		Ricerca 	& \textit{La scelta della staticità va bene, si raggiunge quanto cercato, ma bisognerebbe pensare di posizionarlo in ogni pagina; un'idea è quella di posizionarlo nel menù a destra. Inoltre è da rendere più efficiente con una possibile ricerca vincolata statica.} &	6\\
		Pubblicità 	& \textit{A seguito dei pop-up e delle pagine che si aprono a caso, il voto assegnato scende a 5. Le pubblicità interne non sono troppo invasive e vanno bene in quanto a posizionamento.} &	5\\
		Errore 404	& \textit{Prevedere un testo per ogni lingua sarebbe ottimo. Inoltre giocare sull'accaduto sfruttando una gif legata agli anime e fornire dei consigli per le future ricerche potrebbe far apprezzare maggiormente la pagina. Così fatta non si è stimolati ad effettuare nuove ricerche.}	&	7\\
		\hline
		
	\end{tabularx}
\newpage
	\begin{tabularx}{\textwidth}{|c|X|c|}
		 
		\textbf{Altra pagina del sito} 	& \textit{Per quanto riguarda il contenuto nulla da ridire. Il problema è il design al quale direi di dare una controllada generale non solo alla pagina presa in esame, ma all'intero sito.} & \textbf{7} \\ \hline
		
		\textbf{Mobile} & \textit{Non è possibile che al giorno d'oggi una pagina presenti errori grossolani come quelli qui individuati. Sicuramente la pagina non è stata pensata per il mobile, è stata adattata nel tempo, ma l'apertura di pop-up, uno scroll verticale pressochè infinito e i ridimensionamenti non del tutto corretti, mi hanno portato ad un'esperienza negativa della pagina lato mobile.} & \textbf{4} \\ \hline
		
	\end{tabularx}

\end{center}

In conclusione dalla media dei voti assegnati, in una scala da 0 a 10, per gli aspetti analizzati nel seguente documento, il voto del sito è di: \\
\begin{center}
	\LARGE 6,20
\end{center}
