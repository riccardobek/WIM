\section{Homepage}
Un sito web lo possiamo paragonarlo ad un negozio. Prima di entrare in un nuovo negozio l’approccio comune consiste nel guardare la vetrina. Questo è un punto cruciale, perché é proprio in quel momento che decidiamo se entrare in un negozio oppure proseguire oltre. Questo avviene per il web esattamente allo stesso modo. La homepage in questo contesto é la vetrina del negozio. \\
Arrivati quindi alla vetrina bisogna fornire all'utente nel minor tempo possibile le informazioni che sta cercando o  suscitare la sua curiosità. Il proprietario del sito web deve pertanto cercare di soddisfare le cosiddette 6 W. 


\subsection{I 6 assi informativi}

\subsubsection{Where - "A che sito sono arrivato?"}
Il soggetto ha riconosciuto in tempi non eccessivamente lunghi che il sito si occupa di fornire streaming e download di anime. Ciò è stato possibile grazie alla presenza negli unici testi presenti nella homepage la frase "downolad \& Streaming". In questo modo `e comprensibile a primo sguardo di cosa il sito tratta, per cui se il soggetto fosse stato interessato a guardare una serie animata giapponese avrebbe continuato la navigazione nel sito.
\\
\\
INSERIRE IMMAGINE

\subsubsection{Who - "Chi c'è dietro al sito?"}

\subsubsection{Why - "Perché dovrei dare la mia fiducia? Che benefici mi dà?"}

\subsubsection{What - "Che cosa offre il sito?"}

\subsubsection{When - "C'è qualche novità?"}

\subsubsection{How - "Come faccio a fare le cose?"}

