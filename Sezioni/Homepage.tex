\section{Homepage} \label{Homepage}
Un sito web lo possiamo paragonarlo ad un negozio. Prima di entrare in un nuovo negozio l’approccio comune consiste nel guardare la vetrina. Questo è un punto cruciale, perché é proprio in quel momento che decidiamo se entrare in un negozio oppure proseguire oltre. Questo avviene per il web esattamente allo stesso modo. La homepage in questo contesto é la vetrina del negozio. \\
Arrivati quindi alla vetrina bisogna fornire all'utente nel minor tempo possibile le informazioni che sta cercando o  suscitare la sua curiosità. Il proprietario del sito web deve pertanto cercare di soddisfare le cosiddette 6 W. 

\subsection{I 6 assi informativi} \label{Assi informativi}

\subsubsection{Where - "A che sito sono arrivato?"} \label{HWhere}
Appena arrivato alla seguente pagina, ho riconosciuto in tempi non eccessivamente lunghi che il sito si occupa di fornire streaming e download di anime. Ciò è stato possibile grazie alla presenza negli unici testi presenti nella homepage la frase "downolad \& Streaming". In questo modo é comprensibile a primo sguardo di cosa il sito tratta, per cui se una qualunque persona fosse interessata a guardare una serie animata giapponese potrebbe continuare la navigazione nel sito.
\\
\\
INSERIRE IMMAGINE

\subsubsection{Who - "Chi c'è dietro al sito?"} \label{HWho}
Il sito è facilmente comprensibile che è riferito agli anime. Chiunque riconoscerebbe le immagini e le ricondurrebbe alla categoria generica dei cartoni animati. I tempi di comprensione da questo punto di vista sono molto rapidi. La specifica di essere giapponesi è comprensibile da una parte del nome del sito (Anime), ma soprattutto dai titoli dei cartoni in uscita e recenti presentati attraverso uno slideshow.

\subsubsection{Why - "Perché dovrei dare la mia fiducia? Che benefici mi dà?"} \label{HWhy}
La pagina non presenta alcuna descrizione della sua offerta, procede con il piazzare una lista degli anime del momento e di quelli passati. Presuppone che l'utente sia arrivato alla pagina con l'intento di guardarsi un anime, o in generale ottenere delle informazioni su di esso. Questa comportamento da parte degli ideatori del sito non attira l'utente e non fornisce alcun elemento a cui poter dare fiducia. Sebbene vi siano gli utenti che tornano alla pagina, il numero sarà sicuramente minore rispetto a quello a cui potrebbero aspirare con una descrizione nella homepage e un layout più snello di contenuti rispetto a come è adesso (vedi figura ...).
\\
\\
(IMMAGINE DELLA PARTE DOPO UNO/DUE SCROLL DELLA HOME PAGE)


\subsubsection{What - "Che cosa offre il sito?"} \label{HWhat}
Supponendo di aver attirato l'attenzione dell'utente; procede con effettuare degli scroll e rientrando nella dimensione di circa 2,5 pagine ha immediatamente un effetto positivo sulla pagina. Il contenuto è composto da immagini preview delle serie animate contenute nel sito in ordine di uscita, dai recenti a quelli passati. 

\subsubsection{When - "C'è qualche novità?"} \label{HWhen}
La pagina risponde molto bene a questa domanda. Nella homepage sono presenti gli ultimi articoli inseriti e ad ogni rilascio di un nuovo episodio è possibile scoprire i contenuti nuovi.
Questo aspetto è molto importante perché permette agli utenti, soprattutto coloro che rientreranno più volte nel sito, di rimanere aggiornati e volendo anche poter ampliare le loro conoscenze su altre serie oltre a quelle che lo hanno portato a navigare sul sito \nomeSito. La manutenibilità del sito dal punto di vista delle novità è sicuramente un punto a favore che gli permette di ottenere fiducia da parte degli utenti abituali e occasionali.


\subsubsection{How - "Come faccio ad arrivare alle sezioni principali?"} \label{HHow}
Per quanto riguarda l' "How" del sito possiamo dire che è facile ed intuitivo; utilizza il menù a tendina per la prima voce a menù (vedi figura ...) e le altre due voci, \textit{Contact} e \textit{Chat}, sono semplici link alle pagine di riferimento.
Queste voci sono posizionate in alto alla pagina e l'aspetto negativo è sicuramente che dopo aver effettuato degli scroll verso il basso spariscono. L'utente è costretto ad essere posizionato all'inizio della pagina per poter navigare usando il menù.
Oltre al menù un altro aspetto legato al come arrivare alle principali sezioni è fornito dalla presenza di una modalità di ricerca in alto a destra e dalla seconda modalità basata sui tag a fondo pagina che risulta essere però negativa, come andremo a vedere nella sezione Ricerca (\hyperref[Ricerca]{Sezione \ref{Ricerca}}).






