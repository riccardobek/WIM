\section{Mobile}
Un sito web al giorno d'oggi deve essere pronto ad adattarsi ai diversi schermi. In questo caso andando a ridurre la dimensione della schermata, le pagine del sito iniziano ad avere problemi. Il link che collega alla pagina di Facebook non si ridimensiona correttamente(Figura ...). il resto del contenuto si adatta diminuendo le colonne e rendendo più grandi i blocchi relativi agli ultimi arrivi. La pubblicità viene inserita a fondo pagina. Non vi sono movimenti orizzontali, tuttavia la dimensione verticale inizia ad essere notevole. \\
Il menù e il box di ricerca diventano un problema. La ricerca idealmente dovrebbe trovarsi nel menù della pagina oppure all'inizio, inqueso caso la troviamo dopo il contenuto della pagina e prima della pubblicità dopo diversi scroll verticali che se anche sono apprezzati nei telefonini, sono comunque eccessivi. Per quanto riguarda il menù fa apparire due box uno funzionante, l'altro no, che dovrebbero essere usati per andare alle altre pagine del sito.